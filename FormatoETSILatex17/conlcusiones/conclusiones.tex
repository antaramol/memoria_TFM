\chapter{Conclusiones}\label{chp-04}

% quote from Einstein
\epigraph{If we knew what it was we were doing, it would not be called research, would it?}{Albert Einstein}

Una vez finalizado el proyecto, es momento de hacer una valoración del mismo. 
En este capítulo se exponen las conclusiones obtenidas tras la realización del proyecto, así como las posibles mejoras que se podrían realizar en un futuro.

\section{Hitos conseguidos}
Después de haber estado enfocado en un proyecto exigente durante un periodo de tiempo prolongado, la sensación suele ser de satisfacción.
Pensar que se podrían haber realizado más pruebas, o que se podría haber mejorado algún aspecto, es inevitable.

Sin embargo, hay que echar la vista atrás y ver de dónde venimos para entender el camino recorrido.
El objetivo principal del proyecto, que era crear y desplegar un modelo clasificador de audios ha sido cumplido.

Al no haber utilizado los requisitos de forma rigurosa durante el desarrollo del proyecto, es difícil determinar el grado de éxito del mismo.
Se van a enumerar los requisitos que han sido cumplidos:

\begin{itemize}
    \item El modelo es capaz de clasificar audios por emociones con una precisión superior al 70\%.
    \item El sistema es accesible desde cualquier navegador web.
    \item El modelo puede ser reentrenado con nuevos datos, o entrenar un nuevo modelo.
    \item Todo el código ha sido desarrollado utilizando librerías de código abierto.
    \item La utilización de contenedores permite desplegar el sistema en cualquier máquina que tenga instalado Docker.
    \item La implementación web ha sido implementada sobre HTTPS.
\end{itemize}

Algunos requisitos como los tiempos de respuesta del sistema dependen en gran medida de la capacidad de computación.
Además, pueden ser variables dependiendo de la carga de trabajo del servidor, de la calidad de la red, etc.
Por lo tanto, se necesitaría un estudio más profundo para medirlos correctamente.

\section{Líneas futuras}

\section{Lecciones aprendidas}