% !TEX root =../LibroTipoETSI.tex
\chapter*{Resumen}
\pagestyle{especial}
\chaptermark{Resumen}
\phantomsection
\addcontentsline{toc}{listasf}{Resumen}

\lettrine[lraise=-0.1, lines=2, loversize=0.2]{E}{ste} trabajo se centra en la creación y despliegue de un modelo clasificador de audios por emociones.
La idea surge de un proyecto pasado cuya finalidad era detectar emociones en un caso de uso concreto, a través de llamadas telefónicas.

El modelo se ha implementado en Python, utilizando la librería  \textit{Transformers} de \textit{HuggingFace} y el modelo preentrenado \textit{Wav2Vec2} de \textit{Facebook AI}.
Ha sido desplegado como aplicación web, utilizando \textit{Flask} y \textit{Docker}.

%La hoja de estilo utilizada es una versión de la que el Prof. Payán realizó para un libro que desde hace tiempo viene escribiendo para su asignatura. Con ella se han realizado estas notas, a modo de instrucciones, añadiéndole el diseño de la portada. El diseño de la portada está basado en el que el prof. Fernando García García, de nuestra universidad, hiciera para los libros de la sección de publicación de nuestra Escuela.


\chapter*{Abstract}
\pagestyle{especial}
\chaptermark{Abstract}
\phantomsection
\addcontentsline{toc}{listasf}{Abstract}

\lettrine[lraise=-0.1, lines=2, loversize=0.2]{T}{his} work focuses on the creation and deployment of an audio emotions classifier model.
The idea comes from a past project whose purpose was to detect emotions in a specific use case, through phone calls.

The model has been implemented in Python, using the \textit{Transformers} library from \textit{HuggingFace} and the pre-trained \textit{Wav2Vec2} model from \textit{Facebook AI}.
It has been deployed as a web application, using \textit{Flask} and \textit{Docker}.

\endinput
